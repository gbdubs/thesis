\chapter*{Acknowledgments}

When I asked Professor Storer to supervise me on a thesis on the graph isomorphism problem, he was hesitant.
He only acquiesced when I persuaded him that my future was secure with a fantastic job, and that my primary objective was the pursuit of questions that were of nothing but personal interest to me.
In many respects, his skepticism proved well founded.

This work has been incredibly challenging, both in that the body of existing work on GI is so large, and in that few visible niches of it exist which are promising and not thoroughly explored.
Over the past year I have poured my time and energy into this project, and have found it unbelievably energizing to do so.
I have been thrilled to find interesting properties in problems surrounding GI, and have had an equal number of frustrations in finding that my results had been previously discovered.
Moreover, it has been illuminating to begin to understand a hidden world of graph theory which had before seemed either trivial or intractably complex.

I would like to thank Professor Storer for the initial bout of skepticism about this project, as it shaped this project and experience for the better.
But I would also like to thank him for the amount of advice and freedom that he has supported me with on this project.
It has kept me on track to focus on my real goal for the semester, which was to learn and grow.
I have learned advanced techniques in GPU calculation, proof techniques in abstract algebra, and gotten the chance to reason with established problems in new and interesting ways.
My skill set has been broadened by this project which has deeply challenged me and always kept me on my toes.

I would like to thank my advisors (Prfs. Storer, Di Lillo and Torrey), the Computer Science department, and my friends and family for all of the different kinds of support and encouragement that have brought me to successful completion of this project.

\chapter*{Abstract}

Given a graph, count the number of closed walks (called cycles) of every length that pass through each vertex in the graph.
Counting the number of cycles gives us a vector which describes a kind of local resonance, a description of the localized area around each vertex within the graph.
This idea is a numerical property, called an invariant, which is highly information dense--able to detect differences between graphs (or vertices) with high probability, but not sufficient to verify that two graphs (or vertices in a single graph) are the same.
It turns out that we can calculate the number of Cycles very quickly, relative to how much information it gives us.

The number of graphs you can create, even over a small number of vertices, is very large.
However, the number is \emph{much} larger if you consider different labelings of the same graph to be distinct.
This difference matters in many contexts, but is made clear through examining standard random graph generators, which treat different labelings of the same graph as if they are different graphs. 
This is not a problem in and of itself (it makes sense in many practical contexts), but it warps theoretic arguments about the runtime of algorithms over `random' graphs, in ways that we can describe through probability, algebraic proof and experimental results.
It turns out that making up our own random graph generators can actually improve upon this state of affairs in a quantifiable way.

\chapter*{An Open Source Project}

An interesting aspect of this thesis has been that I have placed every incremental iteration of my work online, through the git version control system and GitHub.
Every element, from my reading notes, to my mid-semester reports, to my code, results, and datasets: everything has been kept in a central repository, and every change has been committed and logged.
Alongside this choice, I have decided that every line of code I have written is open source and publicly available for use; licensed under the Creative Commons Attribution-ShareAlike 4.0 International License.

There are three reasons to make this thesis as an open source, version controlled work.
\begin{enumerate}
\item{
Research in academia is far too often done in the dark, and only once the conclusion has been drawn and sufficiently polished does the general public get to learn about it. 
This falsely represents scientific inquiry as a lightning bolt, a blinding and stark progression from correct idea to correct idea. 

In reality we all explore ideas that fail, we all have intuitions that turn out to be incorrect.
The reality of research is much more one of lightning's infinitesimally branching electrical charges (with eventual connection and brilliance). Though this report presents a well coordinated and rehearse set of conclusions, this is not a reflection of all of the work that was done. The GitHub source provides the other pieces of this puzzle.
}
\item{I work on a personal laptop, a school desktop, and occasionally a public workstation. Transferring data (of all sorts) between these computers is tiresome and error prone without a VCS.}
\item{Everyone does better work when they know that their work could be observed.}
\end{enumerate}

The work is available \href{http://www.github.com/gbdubs/thesis}{on my personal github page}, and will be there for the foreseeable future.