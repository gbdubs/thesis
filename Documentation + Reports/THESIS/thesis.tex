\documentclass[11pt,a4paper]{report}
\usepackage{graphicx}
\begin{document}
\begin{titlepage}
	\centering
	\includegraphics[width=0.15\textwidth]{brandeis-seal}\par\vspace{1cm}
	{\scshape\LARGE Brandeis University \par}
	\vspace{1cm}
	{\scshape\Large Undergraduate Thesis in Computer Science\par}
	\vspace{1.5cm}
	{\huge\bfseries Graph Isomorphism, Reconstruction and the Cycles Invariant\par}
	\vspace{2cm}
	{\Large\itshape Grady Ward\par}
	\vfill
	supervised by\par
	Prf. James \textsc{Storer}
	\vfill

	{\large \today\par}
\end{titlepage}

\tableofcontents

\chapter{Definitions and Syntax}

\section{Graphs}
\subsection{Representations, Labeling, Matrices}

\section{Graph Isomorphism and Automorphism}
\subsection{Graph Invariants}
\subsection{Vertex Invariants}
\subsection{Discriminatory Power}
\subsection{Automorphism Groups}

\section{Cycles Invariant}
\subsection{Running Time}
\subsection{Dealing with Large Numbers}
\subsection{Asymptotic Bit Growth}

\section{Reconstructability, Determined, Representation}



\chapter{Cycles as a Graph Invariant}

\section{Basic Cycles-Reconstructable Properties}
\subsection{Vertices, Edges, Degree Sequence}
\subsection{Chromatic Polynomial}

\section{Other Forms of Reconstructability}
\subsection{EA Reconstructability}
\subsection{Deck Reconstructability}

\section{Placing Cycles within a Time/Power Tradeoff}

\section{Discrimination  on Tough Graph Classes}
\subsection{Background}
\subsection{1-Sparse Graphs}
\subsection{2-Dense Graphs}
\subsection{Miyizaki Graphs}

\section{Imperfection, Co-Cycles Graphs}
\subsection{Discovering Co-Cycles Graphs}
\subsection{Constructing Co-Cycles Graphs}
\subsection{As a Proposed Dataset for Invariant Analysis}

\section{Discriminatory Agreement By N and M}
\subsection{Expectations Borne out of Graph Counts}
\subsection{An Unexpected Dip}




\chapter{Cycles as a Vertex Invariant}

\section{Quantifying how Discrimination Varies with P}
\subsection{Limitations of Cycles' Discriminatory Power}
\subsection{Observational Data}
\subsection{Theoretical Explanation: Path vs Cycle Graphs}

\section{Automorphism 'Quazi-Equivalence Classes'}
\subsection{Background}
\subsection{Vertex Similarity is Transitive}
\subsection{Internal Structure of QEC's}

\section{Improving upon QECs}
\subsection{Appending a Flag, Somewhat Predictable}
\subsection{Theoretical Justification for Flagging}
\subsection{Analytical Support for Flagging}

\section{Limitations to Augmentation}
\subsection{A Second Augmentation Hypothesis}


\chapter{Cycles and the Reconstruction Conjecture}

\section{Reconstruction Conjecture}
\subsection{Background}
\subsection{Manual Verification}
\subsection{Novel Manual Verification}

\section{Cycles of a Deck}
\subsection{The Triangle Identity}
\subsection{Further Identities}
\subsection{Translation to Satisfiability}

\section{If the Reconstruction Conjecture is True}
\subsection{Natural Use of Induction}
\subsection{Using Cycles to Reduce Induction}
\subsection{Using Triangle Identity to Limit Isomorphism Tests}
\subsection{An Asymptotically Fast Algorithm}
\subsection{Further Lines of Exploration}



\chapter{Canonical Labeling Using Cycles}
\section{Background}
\section{A Consistent Algorithm}
\section{Time Growth Comparisons to Faster Algorithms}



\chapter{Random Graph Generators and Automorphisms}

\section{On the Number of Graphs of a given Size}
\subsection{Proposed Closed Forms}

\section{Graphs as Singular Objects and their Multiple Representations}
\subsection{Representations per Graph}
\subsection{Distribution of Representations}

\section{Dominant Random Graph Models}
\subsection{Erdos-Reyni Models}
\subsection{Use and Dominance of Erdos Reyni-Models}

\section{Measuring Flaws of Random Graph Models}
\subsection{Averaging Model}
\subsection{Variance Model}
\subsection{Kurtosis Model}
\subsection{Probability Ratio Model}

\section{Flaws in Proofs of Average Case Random Graphs}
\subsection{Selection of Certain Classes of Graphs}
\subsection{Examples Uses in Proofs}
\subsection{Average Case Runtime Under Erdos-Reyni and Worst Case}

\section{Alternative Ideas for Random Graph Modeling and Creation}
\subsection{Distributional Goals}
\subsection{Cloning Model}

\section{Inherent Limitations on Models by Computational Theory}
\subsection{Uncertainty in Set Size}
\subsection{Computational Verification Limits}



\chapter{Reflections}
\section{Broad Project, Unclear Aims}
\section{Modes of Discovery}
\section{Freedom to Pursue Interest}
\section{Acknowledgments}


\end{document}
