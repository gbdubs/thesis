\documentclass{article} 
\begin{document} 

\textbf{Claim:} 

The problem of Bipartite Isomorphism is GI-complete.

\textbf{Proof:}


It is a well established result that the classes of connected graphs and the classes of regular
graphs are GI-complete.  We can conclude without much difficulty that the class of connected
regular graphs are also GI-Complete. We will be demonstrating that the hardest case of
regular connected isomorphism can be solved in polynomial time relative to the problem of
Bipartite Graph Isomorphism.

The hardest problems in GI-RC are ones where each of the nodes of the graph are
half-connected.  Since it is generally agreed upon that information sparse graphs are easier
to solve for, and we know that if we have two graphs G and H, we can find their isomorphisms
by inverting them (G') and (H'), thus we can always reduce the number of edges in the graph
to half of the maximum or fewer. Thus, the hardest such problem is one that has exactly half
the number of possible edges.  Since we are assuming a regular, connected graph, this
determines the number of edges in the graph if we know the vertices. Assume we have a graph 
\(G\), with \(\#(V) = N\) vertices, then the number of edges is simple: 
\[\#(E) = M = \frac{V^2 - V}{4}\]

Since we are assuming that our graph is regular connected and that each vertex is connected to half
of the other vertices in the graph, we have imposed the constraint that the diameter of the graph
is exactly two. This can be proven easily. Imagine that the diameter of the graph is 3 or more.
That means there exist two nodes such that the shortest path between them is of size 3 or more.
Call these hypothetical nodes \(a\) and \(b\).  If we examine the adjacent sets \(A = \{x \in V \| adj(a, x)\}\), and
\(B = \{x \in V \| adj(b, x)\}\), we know that the size of each must be exactly the degree of \(a\) and
\(b\), which as the shared degree of the graph (as discussed above) must be \(\frac{V-1}{2}\).
The sets A and B must be disjoint (otherwise there would exist a path of length 2 between \(a\)
and \(b\).  However, if \(A\) and \(B\) are disjoint, that means that the size of the vertex set V
is exactly: 
\[V = N \geq \#(A) + \#(B) + \#(a) + \#(b)\]



Now imagine that we have a magical algorithm, \(A(G)\) which can solve GI in polynomial
time given that the graph is a bipartite graph.  We will use this hypothetical algorithm to show
that the class of bipartite graphs is GI complete, ans we can use \(A(G)\) to solve GI for the
set of all connected regular graphs (referred to from here on out as CR Graphs.



You are given the problem of determining whether or not two CR graphs are isomorphic.
This algorithm will do so in polynomial time, given \(A(G)\).  Lets call these two graphs 
Y and Z.  Select a vertex at random from Y and call it y.  Select a vertex random from Z
and call it z.  Note that we will be holding y constant while we rotate through z.  Obviously,
if \(y \arrow z\) at any point during our algorithm, then we can return and say that we have
found an isomorphism between Y and Z.

\end{document}